% Autogenerated translation of about.md by Texpad
% To stop this file being overwritten during the typeset process, please move or remove this header

\documentclass[12pt]{book}
\usepackage{graphicx}
\usepackage{fontspec}
\usepackage[utf8]{inputenc}
\usepackage[a4paper,left=.5in,right=.5in,top=.3in,bottom=0.3in]{geometry}
\setlength\parindent{0pt}
\setlength{\parskip}{\baselineskip}
\setmainfont{Helvetica Neue}
\usepackage{hyperref}
\pagestyle{plain}
\begin{document}

\hrule
permalink: /
title: "My research"
excerpt: "About me"
author\emph{profile: true
redirect}from: 
  - /about/

\section*{  - /about.html}

I specialize in extreme value theory, heavy-tailed time series, and machine learning. My work often involves risk assessment with a primary focus on climate science applications.

\section*{Ongoing projects}

Currently learning about out-of-domain generalization problems in machine learning.

\section*{Short bio}

I'm a lecturer in statistics at MIA Paris-Saclay, AgroParisTech. Previously, I was a postdoctoral researcher at the \href{https://www.unige.ch/gsem/fr/recherche/corps-professoral/chercheurs/gloria-buritica-borda/}{Université de Genève} working with Prof. \href{http://www.sengelke.com}{Sebastian Engelke}. I completed my Ph.D. at the LPSM laboratory in Sorbonne Université under the supervision of Prof. \href{https://www.lsce.ipsl.fr/Phocea/Pisp/visu.php?id=176&uid=naveau}{Philippe Naveau} and Prof. \href{http://wintenberger.fr}{Olivier Wintenberger}. 

\href{/files/CV_BURITICA.pdf}{\emph{curriculum vitae}}, 
\href{mailto:gloriapatricia.buriticaborda@agroparistech.fr}{\emph{e-mail}}.

Beyond my research, I love reading and a fresh just-grind cup of Colombian coffee. I specially enjoy gathering with friends and family around flavorful food and pursuing bucolic landscapes. 

\subsection*{Ph.D. Thesis}

Here you can find the \href{/files/Oral_slides-10.pdf}{\emph{slides}} and the final \href{/files/these_archivage_3701601.pdf}{\emph{manuscript}} of my Ph.D. thesis entitled "Assessing the time dependence of multivariate extremes for heavy rainfall modeling”.

\end{document}
